\section{Exigences non fonctionnelles}
    \subsection{Exigences du produit}
        \subsubsection{Exigences de portabilité}
            \begin{itemize}
            \renewcommand\labelitemi{\textbullet}
                \item Le programme doit pouvoir s'exécuter sur les ordinateurs des salles machines, plus connu sous le nom de PaDi, Parc Didactique informatique de l'Université Libre de Bruxelles.
            \end{itemize}
        \subsubsection{Exigences de fiabilité}
            \begin{itemize}
            \renewcommand\labelitemi{\textbullet}
                \item En cas d'interruption abrupte d'un duel, les joueurs doivent pouvoir récupérer les informations de celle-ci.
                \item La database doit pouvoir être sécurisé et les utilisateurs ne peuvent y avoir accès sans passer par le serveur.
            \end{itemize}
    \subsection{Exigences d'organisation}
        \subsubsection{Exigences d'implémentation}
            \begin{itemize}
            \renewcommand\labelitemi{\textbullet}
            \item Le programme doit être implémenté en C++14 et doit pouvoir être compilé à l'aide de G++--5.
        \end{itemize}
    \subsection{Exigences externes}
        \subsubsection{Exigences sur la legislation}
        \begin{itemize}
            \renewcommand\labelitemi{\textbullet}
            \item Le système s'engage à ne fournir aucunes informations personnelles à propos des utilisateurs excepté leur nom d'utilisateur. 
        \end{itemize}
        \subsubsection{Exigences de l'interopérabilité}
        \begin{itemize}
            \renewcommand\labelitemi{\textbullet}
            \item Deux joueurs doivent pouvoir s'affronter et communiquer indépendamment de la machine sur laquelle le jeux s'exécute pour autant que celle-ci respecte les exigences du produit.
        \end{itemize}