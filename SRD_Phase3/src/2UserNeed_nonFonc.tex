\section{Exigences non fonctionnelles}

\iffalse
\subsection{Product Requirements}
        \subsubsection{Efficiency Requirements}
            \begin{itemize}
            \renewcommand\labelitemi{\textbullet}
                \item No specific performance requirement has been given, such as an specific FPS (Frames Per Second) at which the program must run for example, but the main performance issue to overcome will be the proper synchronisation of images and animations in the graphical interface with what is actually happening real-time at each client machine connected to the server.
                \item In the same manner as the performance requirements, the program is designed for an specific platform with a high performance and hence space constraints are rather \textit{common sense}, in the sense that the size must be justified by the programs capabilities, as opposed to a delimited space that must be kept in mind throughout the project (such as is the case in mobile phone applications with a very limited memory).
            \end{itemize}
\fi
    \subsection{Exigences du produit}
    \subsubsection{Exigences d'efficacité}
        \begin{itemize}
        \renewcommand\labelitemi{\textbullet}
            \item L'interface\index{interface} graphique doit être parfaitement synchronisée en temps réel pour chaque client\index{client} en fonction des actions individuelles de ceux-ci.
        \end{itemize}
\iffalse
        \subsubsection{Portability Requirements}
        The only requirement in terms of portability is its use across any machine in the \textbf{Salle PC} of the \textbf{PaDi}, otherwise know as the \textbf{Pa}rc \textbf{Di}dactique informatique de l'Universite Libre de Bruxelles
\fi

\iffalse
        \subsubsection{Reliability Requirements}
        \begin{itemize}
        \renewcommand\labelitemi{\textbullet}
            \item A user must not have to be careful about his input into the program, the program itself must be able to manage invalid user input through the means of messages to said user without calling for a termination of the program.
            \item During a connection the server should not turn off. Measures have to be put into place to ensure this critical requirement is met.
            \item As with the reliability of the server, the databases (User and Card ones) utilized by the program must have a backup mechanism so as to ensure that there is no loss of data.
            \end{itemize}
\fi
    \subsubsection{Exigences de fiabilité}
        \begin{itemize}
        \renewcommand\labelitemi{\textbullet}
        \item Une commande incorrecte introduite par l'utilisateur\index{utilisateur} doit pouvoir lui être signalé sans interruption du programme.
        \end{itemize}
\iffalse
        \subsubsection{Usability Requirements}
        Given the learning curve this type of game has, its usability is rather varied:
        \begin{itemize}
        \renewcommand\labelitemi{\textbullet}
            \item Starting to play is rather simple and intuitive; I have $n$ energy, this card costs $m$ energy, hence I now know if i can use it immediately.
            \item Getting better at the game on the other hand involves a more human reasoning and intuition as to possible actions of the adversary, efficiency of high cost/ high damage against low cost/low damage and many other small subtleties that make the difference between a player, and a good player.
        \end{itemize}
        The main concern with usability will therefore be a simple introductory tutorial into the mechanics of the game by means of a PDF for example, and then it will be up to the user to play as much as he came so as to have more data to be able to be a better judge of how to play. A more separate \textit{hands off} approach is to be taken therefore.
\fi
    \subsubsection{Exigences d'ergonomie}
        \begin{itemize}
        \renewcommand\labelitemi{\textbullet}
        \item L'interface\index{interface} graphique doit être orgranisée et facile d'utilisation.
        \item De la documentation pour utilisateur doit être disponible.
        \item La prise en main du système\index{système} doit être abordable.
        \item Les messages d'erreur doivent indiqués comment résoudre ceux-ci.
        \item Un tutoriel d'introduction est mis à disposition du joueur débutant.
        \end{itemize}
        
    \subsection{Exigences d'organisation}
    \subsubsection{Exigences de livraison}
    \begin{itemize}
        \renewcommand\labelitemi{\textbullet}
        \item Le programme complet doit être livré au plus tard pour le 25 mars 2016.
    \end{itemize}

\iffalse
    \subsection{Organisational Requirements}
        \subsubsection{Implementation requirements}
        \subsubsection{Standards Requirements}
        \subsubsection{Delivery Requirements}
\fi
    \subsection{Exigences externes}
    \subsubsection{Exigences de l'éthique}
    \begin{itemize}
        \renewcommand\labelitemi{\textbullet}
        \item Aucun propos diffamatoires ou racistes ne peut être tenus sur le chat du jeu sous peine d'exclusion.
    \end{itemize}