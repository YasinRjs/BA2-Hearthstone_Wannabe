\chapter{Introduction}
\section{But du projet}
Ce document a pour but de présenter, dans le cadre d'une deuxième année de Bachelier en Sciences informatiques, une description détaillée de notre projet multidisciplinaire. Le but, les caractéristiques, l'interface\index{interface} du système\index{système} ainsi que les contraintes auxquelles le programme est soumis sont expliqués. Ce présent rapport est destiné aussi bien au client qu'au développeur soucieux de comprendre l'architecture du programme.
\section{Cadre du projet}
Le programme est un jeux nommé Wizard-Poker, un jeu de carte fantastique où deux utilisateurs\index{utilisateur} s'affrontent dans un duel. Chaque utilisateur possède une collection\index{collection} de carte avec laquelle il peut créer des \index{deck}decks, paquet de vingt cartes avec lequel il commence un duel. Ces cartes sont uniquement de type créature.\\
Au début d'un duel, chaque utilisateur possède un maximum de vingt points de vie \index{points de vie} appelé PV et 3 cartes en main piochées au préalable de son deck. À chaque début de tour,le joueur concerné pioche une carte et gagne un point d'énergie\index{points d'énergie}. Durant son tour, un joueur peut invoquer une créature et/ou attaquer avec une de ses créatures sur le plateau de jeu. Un utilisateur gagne le duel une fois que son adversaire abandonne ou tombe à 0 points de vie.