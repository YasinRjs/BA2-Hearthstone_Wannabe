\documentclass{article}
\usepackage[utf8]{inputenc} 
\usepackage[T1]{fontenc}
\usepackage[french]{babel}
\title{SRD}
\author{Groupe 6}
\date{}
\begin{document}
\maketitle
\section{Général}
J'ai refait un un sharelatex à partir de zéro car il y a énormément de chose à changer de l'ancien. Je pense que l'on ira plus vite et qu'on fera moins d'erreurs en partant de rien qu'en reprenant l'ancien. (mais vous pouvez reprendre des parties si besoins)\\
Tout les schémas sont à refaire. N'oubliez pas que les schémas doivent parler d'eux mêmes.(le lecteur devrait comprendre sans aucun texte en dessous)\\
On essayera d'utiliser le plus possible les termes francais puisque on a commencé à coder le jeu en francais (essayer de garder une logique)\\
Essayer d'avoir une bonne compréhension du code avant de faire votre partie afin de se baser le plus possible sur celui-ci.\\
Envoyer moi un message (fb/gsm) qd vous avez fini votre partie, je relirai chaque partie pour l'orthographe et pour être sur que ça s'insére dans le SRD avec les autres.
\section{Conventions}
\begin{itemize}
\item Les termes relatifs à un concept du projet doivent être impérativement des termes présents dans le glossaire (rajouter si pas encore présent). Si vous voulez utiliser le même mot pour désigner un concept différent, changer! (et ajouter le dans le glossaire). Pas d'utilisation de la traduction francaise d'un concept, si on l'utilise en anglais, on l'utilise en anglais partout!
\item pas de indent au début des paragraphes
\item pas d'espace avant la virgule, un espace après
\item chaque "section" a un tex file correspondant. Si on ajoute une "section", ajouter un tex file.
\item si vous pensez qu'un mot à sa raison d'être dans l'index vous mettez "example(backslash)index\{example\}".
\end{itemize}

\section{Répartition des tâches}
\begin{itemize}
\item intro: sacha
\item glossaire: tt le monde (déjà été arrangé, juste rajouter terme si besoin)
\item historique: sacha
\item UserNeed / besoins fonc.: Youcef/Jackie/Sitil , n'oubliez pas que ça doit être dans un langage compréhensible du client. Vous pouvez vous baser sur ce que Miguel avait fait. Oubliez d'itentifier les acteurs et use cases. Oubliez les 3 relations d'un Use case, à savoir généralisation, inclusion, extension.
\item UserNeed / besoins non fonc: Miguel (cf slide 234 cours analyse et méthode pour voir les différents type d'exigence non fonc)
\item UserNeed / besoins du domaine: Sacha
\item SystNeed / besoins fonc.: Youcef/Jackie/Sitil, cette partie doit absolument se baser complètement sur les UserNeed , sauf qu'ici on décrit avec des termes plus techniques mais on reste dans le "What" et pas le "How".
\item SystNeed / besoins non fonc.: Miguel , idem que pour les besoins fonc.
\item SystNeed / Design : Yasin , le diagramme de classe , essaye de le faire plus long que large pour que ça tienne sur une page a4 
\item SystNeed / diagramme d'activité client-server: Jalal
\end{itemize}
Finissez votre partie pour dimanche soir. Il y aura encore peut-être qlq diagramme à rajouter, on aura encore qlq jours pour y penser. Rdv lundi à 12h pour premier debriefing SRD.
\end{document}