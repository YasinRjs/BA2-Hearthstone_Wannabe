\documentclass{article}
\usepackage[utf8]{inputenc} 
\usepackage[T1]{fontenc}
\usepackage[french]{babel}
\title{Debriefing 1}
\author{Groupe 6}
\date{}
\begin{document}
\maketitle
\section{Remarques SRD}
\begin{itemize}
\item De manière général, un gros manque d'homogénéité, c'est à dire différentes conventions à travers le rapport, différents termes pour désigner la même chose, ...
\item Exigences du domaine: doit être refait, il y a des contraintes à trouver (même bidon)
\item Exigences non-fonctionelles: on doit mettre qlch!
\item diagramme UML: enlever la flèche oblique, account n'est pas un objet (un objet peut et va être détruit quand on quitte le jeu), ici on cherche bien à avoir un diagramme détaillé du code (devrait être plus claire mnt qu'on a coder)
\item diagramme "lancement du jeu": complètement inutile (peut être interessant de faire un diagramme d'activité client-serveur à la place)
\item diagramme "gestion des événements": bcp trop chargé , on doit le refaire en épurant ou en fesant plusieurs sur des parties plus précises
\item diagramme "fin de partie": inutile , on peut le remplacer par 2 phrases pour faire passer cette information.
\end{itemize}

\end{document}